% !TEX root = ./CV_Lasse_D_Skaalum.tex

%============================================================================%
%
%	DOCUMENT DEFINITION
%
%============================================================================%

%we use article class because we want to fully customize the page and don't use a cv template
\documentclass[10pt,A4]{article}	

% Skabelon: https://www.overleaf.com/latex/templates/jan-kusters-left-sidebar-cv/tmmnhrkcmpgv

%----------------------------------------------------------------------------------------
%	ENCODING
%----------------------------------------------------------------------------------------

% we use utf8 since we want to build from any machine
\usepackage[utf8]{inputenc}

%----------------------------------------------------------------------------------------
%	LOGIC
%----------------------------------------------------------------------------------------

% provides \isempty test
\usepackage{xstring, xifthen}

%----------------------------------------------------------------------------------------
%	FONT BASICS
%----------------------------------------------------------------------------------------

% some tex-live fonts - choose your own

%\usepackage[defaultsans]{droidsans}
%\usepackage[default]{comfortaa}
%\usepackage{cmbright}
\usepackage[default]{raleway}
%\usepackage{fetamont}
%\usepackage[default]{gillius}
%\usepackage[light,math]{iwona}
%\usepackage[thin]{roboto} 

% set font default
\renewcommand*\familydefault{\sfdefault} 	
\usepackage[T1]{fontenc}

% more font size definitions
\usepackage{moresize}
% to be able to use ½
\usepackage{textcomp}

%----------------------------------------------------------------------------------------
%	FONT AWESOME ICONS
%---------------------------------------------------------------------------------------- 

% include the fontawesome icon set
\usepackage{fontawesome}

% use to vertically center content
% credits to: http://tex.stackexchange.com/questions/7219/how-to-vertically-center-two-images-next-to-each-other
\newcommand{\vcenteredinclude}[1]{\begingroup
\setbox0=\hbox{\includegraphics{#1}}%
\parbox{\wd0}{\box0}\endgroup}

% use to vertically center content
% credits to: http://tex.stackexchange.com/questions/7219/how-to-vertically-center-two-images-next-to-each-other
\newcommand*{\vcenteredhbox}[1]{\begingroup
\setbox0=\hbox{#1}\parbox{\wd0}{\box0}\endgroup}

% icon shortcut
\newcommand{\icon}[3] { 							
	\makebox(#2, #2){\textcolor{maincol}{\csname fa#1\endcsname}}
}	

% icon with text shortcut
\newcommand{\icontext}[4]{ 						
	\vcenteredhbox{\icon{#1}{#2}{#3}}  \hspace{2pt}  \parbox{0.9\mpwidth}{\textcolor{#4}{#3}}
}

% icon with website url
\newcommand{\iconhref}[5]{ 						
    \vcenteredhbox{\icon{#1}{#2}{#5}}  \hspace{2pt} \href{#4}{\textcolor{#5}{#3}}
}

% icon with email link
\newcommand{\iconemail}[4]{ 						
    \vcenteredhbox{\icon{#1}{#2}{#4}}  \hspace{2pt} \href{mailto:#3}{\textcolor{#4}{#3}}
}

%----------------------------------------------------------------------------------------
%	PAGE LAYOUT  DEFINITIONS
%----------------------------------------------------------------------------------------

% page outer frames (debug-only)
% \usepackage{showframe}		

% we use paracol to display breakable two columns
\usepackage{paracol}

% define page styles using geometry
\usepackage[a4paper]{geometry}

% remove all possible margins
\geometry{top=1cm, bottom=1cm, left=1cm, right=1.5cm}

% fixes a wierd bug where items in itemize environments would break the margin
\usepackage{enumitem}
\setlist{rightmargin=2.5mm}

\usepackage{fancyhdr}
\pagestyle{empty}

% space between header and content
% \setlength{\headheight}{0pt}

% indentation is zero
\setlength{\parindent}{0mm}

% Hyphenation
\hyphenation{In-te-res-se-e-le-ment}

%----------------------------------------------------------------------------------------
%	TABLE /ARRAY DEFINITIONS
%---------------------------------------------------------------------------------------- 

% extended aligning of tabular cells
\usepackage{array}

% custom column right-align with fixed width
% use like p{size} but via x{size}
\newcolumntype{x}[1]{%
>{\raggedleft\hspace{0pt}}p{#1}}%


%----------------------------------------------------------------------------------------
%	GRAPHICS DEFINITIONS
%---------------------------------------------------------------------------------------- 

%for header image
\usepackage{graphicx}

% use this for floating figures
% \usepackage{wrapfig}
% \usepackage{float}
% \floatstyle{boxed} 
% \restylefloat{figure}

%for drawing graphics		
\usepackage{tikz}				
\usetikzlibrary{shapes, backgrounds,mindmap, trees}

%----------------------------------------------------------------------------------------
%	Color DEFINITIONS
%---------------------------------------------------------------------------------------- 
\usepackage{transparent}
\usepackage{color}

% primary color
\definecolor{maincol}{RGB}{ 50, 100, 250 }

% accent color, secondary
% \definecolor{accentcol}{RGB}{ 250, 150, 10 }

% dark color
\definecolor{darkcol}{RGB}{ 70, 70, 70 }

% light color
\definecolor{lightcol}{RGB}{245,245,245}

% Package for links, must be the last package used
\usepackage[hidelinks]{hyperref}

% returns minipage width minus two times \fboxsep
% to keep padding included in width calculations
% can also be used for other boxes / environments
\newcommand{\mpwidth}{\linewidth-\fboxsep-\fboxsep}
	


%============================================================================%
%
%	CV COMMANDS
%
%============================================================================%

%----------------------------------------------------------------------------------------
%	 CV LIST
%----------------------------------------------------------------------------------------

% renders a standard latex list but abstracts away the environment definition (begin/end)
\newcommand{\cvlist}[1] {
	\begin{itemize}{#1}\end{itemize}
}

%----------------------------------------------------------------------------------------
%	 CV TEXT
%----------------------------------------------------------------------------------------

% base class to wrap any text based stuff here. Renders like a paragraph.
% Allows complex commands to be passed, too.
% param 1: *any
\newcommand{\cvtext}[1] {
	\begin{tabular*}{1\mpwidth}{p{1\mpwidth}}
		\parbox{1\mpwidth}{#1}
	\end{tabular*}
}

%----------------------------------------------------------------------------------------
%	CV SECTION
%----------------------------------------------------------------------------------------

% Renders a a CV section headline with a nice underline in main color.
% param 1: section title
\newcommand{\cvsection}[1] {
	\vspace{14pt}
	\cvtext{
		\textbf{\huge{\textcolor{darkcol}{\uppercase{#1}}}}\\[-4pt]
		\textcolor{maincol}{ \rule{0.1\textwidth}{2pt} } \\
	}
}

%----------------------------------------------------------------------------------------
%	META SKILL
%----------------------------------------------------------------------------------------

% Renders a progress-bar to indicate a certain skill in percent.
% param 1: name of the skill / tech / etc.
% param 2: level (for example in years)
% param 3: percent, values range from 0 to 1
\newcommand{\cvskill}[3] {
	\begin{tabular*}{1\mpwidth}{p{0.72\mpwidth}  r}
 		\textcolor{black}{\textbf{#1}} & \textcolor{maincol}{#2}\\
	\end{tabular*}%
	
	\hspace{4pt}
	\begin{tikzpicture}[scale=1,rounded corners=2pt,very thin]
		\fill [lightcol] (0,0) rectangle (1\mpwidth, 0.15);
		\fill [maincol] (0,0) rectangle (#3\mpwidth, 0.15);
  	\end{tikzpicture}%
}


%----------------------------------------------------------------------------------------
%	 CV EVENT
%----------------------------------------------------------------------------------------

% Renders a table and a paragraph (cvtext) wrapped in a parbox (to ensure minimum content
% is glued together when a pagebreak appears).
% Additional Information can be passed in text or list form (or other environments).
% the work you did
% param 1: time-frame i.e. Sep 14 - Jan 15 etc.
% param 2:	 event name (job position etc.)
% param 3: Customer, Employer, Industry
% param 4: Short description
% param 5: work done (optional)
% param 6: technologies include (optional)
% param 7: achievements (optional)
\newcommand{\cvevent}[7] {
	
	% we wrap this part in a parbox, so title and description are not separated on a pagebreak
	% if you need more control on page breaks, remove the parbox
	\parbox{\mpwidth}{
		\begin{tabular*}{1\mpwidth}{p{0.72\mpwidth}  r}
	 		\textcolor{black}{\Large\textbf{#2}} & \colorbox{maincol}{\makebox[0.25\mpwidth]{\textcolor{white}{#1}}} \\
			\textcolor{maincol}{\textbf{#3}} & \\
		\end{tabular*}\\[8pt]
	
		\ifthenelse{\isempty{#4}}{}{
			\cvtext{#4}
		}
	}

	\ifthenelse{\isempty{#5}}{}{
		\vspace{1pt} %Her stod 9pt før
		{#5}
	}

	\ifthenelse{\isempty{#7}}{}{
		\vspace{1pt} %Her stod 9pt før
		\cvtext{\textbf{Results:}}
		{#7}
	}

	\ifthenelse{\isempty{#6}}{}{
		\vspace{1pt} %Her stod 9pt før
		\cvtext{\textbf{Technologies:}}
		{#6}
	}
	\vspace{14pt}
}

%----------------------------------------------------------------------------------------
%	 CV META EVENT
%----------------------------------------------------------------------------------------

% Renders a CV event on the sidebar
% param 1: title
% param 2: subtitle (optional)
% param 3: customer, employer, etc,. (optional)
% param 4: info text (optional)
\newcommand{\cvmetaevent}[4] {
	\textcolor{maincol} {\cvtext{\textbf{\begin{flushleft}#1\end{flushleft}}}}

	\ifthenelse{\isempty{#2}}{}{
	\textcolor{darkcol} {\cvtext{\textbf{#2}} }
	}

	\ifthenelse{\isempty{#3}}{}{
		\cvtext{{ \textcolor{darkcol} {#3} }}\\
	}

	\cvtext{#4}\\[14pt]
}

%---------------------------------------------------------------------------------------
%	QR CODE
%----------------------------------------------------------------------------------------

% Renders a qrcode image (centered, relative to the parentwidth)
% param 1: percent width, from 0 to 1
\newcommand{\cvqrcode}[1] {
	\begin{center}
        \textcolor{maincol}{Min LinkedIn}
    \end{center}
	\begin{center}
		\includegraphics[width={#1}\mpwidth]{qrcode}
	\end{center}
}


%============================================================================%
%
%
%
%	DOCUMENT CONTENT
%
%
%
%============================================================================%
\begin{document}
\columnratio{0.32}
\setlength{\columnsep}{2.2em}
\setlength{\columnseprule}{4pt}
\colseprulecolor{lightcol}
\begin{paracol}{2}
	\begin{leftcolumn}
		%---------------------------------------------------------------------------------------
		%	META IMAGE
		%----------------------------------------------------------------------------------------
		\begin{figure}[h]
			\centering
			\includegraphics[width=0.93\linewidth]{pb.jpg}	%trimming relative to image size and aligning "FÆRDIGHEDER" with "PROFIL"
		\end{figure}


		%---------------------------------------------------------------------------------------
		%	META SKILLS
		%----------------------------------------------------------------------------------------
		\cvsection{SKILLS}

		% \cvskill{Undervisningserfaring}
		%         {12 år} {1} \\[-2pt]

		% \cvskill{Samarbejde}
		%         {} {1} \\[-2pt]

		% \cvskill{Kommunikation}
		%         {} {1} \\[-2pt]

		\cvskill{Googling}
		{20 yr} {1} \\[-2pt]

		\cvskill{C\# / ASP.NET}
		{3 yr} {0.9} \\[-2pt]

		\cvskill{React.js}
		{2 ½ yr} {0.9} \\[-2pt]

		% \cvskill{TypeScript}
		% {2 yr} {0.9} \\[-2pt]
		
		% \cvskill{JavaScript}
		% {1 ½ yr} {0.9} \\[-2pt]

		\cvskill{C (ANSI C89)}
		{1 ½ yr} {0.8} \\[-2pt]

		\cvskill{Git}
		{4 yr} {0.8} \\[-2pt]

		\cvskill{React Native}
		{1 yr} {0.7} \\[-2pt]

		% \cvskill{Node.js}
		% {2 yr} {0.7} \\[-2pt]
		
		\cvskill{SQL}
		{2 ½ yr} {0.7} \\[-2pt]

		\cvskill{Azure}
		{1 ½ yr} {0.6} \\[-2pt]
		
		\cvskill{Java}
		{½ yr} {0.5} \\[-2pt]

		\cvskill{Kotlin}
		{½ yr} {0.3} \\[-2pt]

		\cvskill{Swift}
		{½ yr} {0.3} \\[-2pt]

		% \cvskill{Softwareingeniør- studerende}
		% {3 yr} {0.6} \\[-2pt]
		
		% \vfill\null
		% \iconhref{Home}{12}{Skaalum.eu (Portfolio)}{https://skaalum.eu/}{black}\\[6pt]
		\iconhref{Github}{12}{GitHub.com/humleflue}{https://github.com/humleflue}{black}\\[6pt]

		\vfill\null
		\cvsection{KONTAKT}\\
		\icontext{MapMarker}{12}{Strynøgade 5, 1. 05\\2100 Copenhagen, Denmark}{black}\\[6pt]
		\iconhref{MobilePhone}{12}{+45 4241 1996}{tel:+4542411996}{black}\\[6pt]
		\iconemail{Envelope}{12}{lasse@bluebirdday.dk}{black}\\[6pt]
		\iconhref{Linkedin}{12}{linkedin.com/in/lasse-skaalum}{https://www.linkedin.com/in/lasse-skaalum/}{black}\\[6pt]

		% \vfill\null
		% \cvqrcode{0.7}

		%---------------------------------------------------------------------------------------
		%	EDUCATION
		%----------------------------------------------------------------------------------------
		\newpage
		\cvsection{EDUCATION}

		\cvmetaevent
		{2019 - 2022}
		{B. Sc. Software}
		{Aalborg University}
		{\textbf{Course Overview (clickable)}
			\setlist{leftmargin=4mm} %reduces the margin for this itemize to make more space
			\cvlist{
				% 1. semester
				\item \href
				{https://moduler.aau.dk/course/2019-2020/DSNDATFB105}
				{Imperative programming}
				\item \href
				{https://moduler.aau.dk/course/2019-2020/DSNDATFB103}
				{Discrete Mathematics}
				% 2. semester
				\item \href
				{https://moduler.aau.dk/course/2019-2020/DSNDATFB211}
				{Algorithms and Data Structures}
				\item \href
				{https://moduler.aau.dk/course/2019-2020/DSNDATFB212}
				{Internet and Web Development}
				\item \href
				{https://moduler.aau.dk/course/2019-2020/DSNDATFB202}
				{Probability Theory and Linear Algebra}
				% 3. semester
				\item \href
				{https://moduler.aau.dk/course/2019-2020/DSNDATFB311}
				{Object Oriented Programming}
				\item \href
				{https://moduler.aau.dk/course/2019-2020/DSNDATFB312}
				{System Development}
				\item \href
				{https://moduler.aau.dk/course/2019-2020/DSNDATFB313}
				{Design and Evaluation of User Interfaces}
				% 4. semester
				\item \href
				{https://moduler.aau.dk/course/2019-2020/DSNDATFB411}
				{Languages and Compilers}
				\item \href
				{https://moduler.aau.dk/course/2019-2020/DSNDATFB412}
				{Syntax and Semantics}
				\item \href
				{https://moduler.aau.dk/course/2019-2020/DSNDATFB413}
				{Computer Architecture and Operating Systems}
				% 5. semester
				\item \href
				{https://moduler.aau.dk/course/2019-2020/DSNDATFB512}
				{Agile Software Engineering}
				\item \href
				{https://moduler.aau.dk/course/2019-2020/DSNDATFB513}
				{Machine Intelligence}
				\item \href
				{https://moduler.aau.dk/course/2019-2020/DSNDATFB514}
				{Database Systems}
				% 6. semester
				\item \href
				{https://moduler.aau.dk/course/2019-2020/DSNSWB611}
				{Advanced Algorithms and Computability}
				\item \href
				{https://moduler.aau.dk/course/2019-2020/DSNSWB612}
				{Models and Tools for Cyber-Physical Systems}
				\item \href
				{https://moduler.aau.dk/course/2019-2020/DSNSWB613}
				{Security}
			}}
		\setlist{leftmargin=10mm} % reset the margin of the itemize

		% \cvmetaevent
		% {2013 - 2016}
		% {Studentereksamen (stx)}
		% {Silkeborg Gymnasium}
		% {Studerende på linjen: Bioteknologi A, Matematik A, Fysik B.\\
		% 	Tilvalgsfag: Informationstekno-logi C}

		% \cvmetaevent
		% {2012 - 2013}
		% {Efterskole}
		% {Vestbirk Musik- og Sportsefterskole}
		% {Elev på 10. klassetrin med linjerne musik og springgymnastik.}

		% \vfill\null
		% \cvqrcode{0.7}

		%---------------------------------------------------------------------------------------
		%	CERTIFICATION
		%----------------------------------------------------------------------------------------
		% \newpage
		\cvsection{CERTIFICATION}
		
		\cvmetaevent
		{2023}
		{Kubernetes Developer}
		{\href{https://training.linuxfoundation.org/certification/certified-kubernetes-application-developer-ckad/}{Certified Kubernetes Application Developer (CKAD)}}
		{The CKAD can design, build and deploy cloud-native applications for Kubernetes.\\
A CKAD can define application resources and use Kubernetes core primitives to create/migrate, configure, expose and observe scalable applications.}

		% \cvmetaevent
		% {2019}
		% {Højskoleophold}
		% {Forårssemester}
		% {Forårssemester på \textbf{Gerlev Idrætshøjskole} med parkour som hovedfag og springgymnastik samt windsurfing som sidefag.}

		% \cvmetaevent
		% {2018}
		% {Skiinstruktør}
		% {BSI 1}
		% {En uges skiinstruktørundervisning med \textbf{Den Danske Skiskole} og bestod alle mine eksaminer.}

		% \cvmetaevent
		% {2016}
		% {Master Class Matematik}
		% {Master Class Matematik 2016}
		% {Deltog i Master Class Matematik 2016 (\textbf{RSA kryptering}), som er et kursus arrangeret af \textbf{Sciencetalenter} for særligt talentfulde elever i 3.g.}

		% \cvmetaevent
		% {2015}
		% {Parkourtræner}
		% {DGIs parkourtræneruddannelse}
		% {Modtog en uges parkour-trænerundervisning og bestod min eksamen.}

		% \vfill\null
		% \cvqrcode{0.7}

		% \newpage
		% \mbox{} % hotfix to place qrcode on the bottom when there are not other elements
		% \vfill
		% \cvqrcode{0.7}

		\newpage
	\end{leftcolumn}
	\begin{rightcolumn}
		%---------------------------------------------------------------------------------------
		%	TITLE  HEADER
		%----------------------------------------------------------------------------------------
		\fcolorbox{white}{darkcol}{\begin{minipage}[c][3.5cm][c]{1\mpwidth}
				\begin {center}
				\HUGE{ \textbf{ \textcolor{white}{ \uppercase{ Lasse D. Skaalum } } } } \\[-24pt]
				\textcolor{white}{ \rule{0.1\textwidth}{1.25pt} } \\[4pt]
				\large{ \textcolor{white} {Turning coffee into code} }
				\end {center}
			\end{minipage}} \\[14pt]
		\vspace{-12pt}

		\textit{Last updated: \today}

		%---------------------------------------------------------------------------------------
		%	PROFILE
		%----------------------------------------------------------------------------------------
		\vfill\null
		% Make sure no text breaks the margins
		\emergencystretch 3em
		\cvsection{PROFILE}

		\cvtext{I am a structured, ambitious and curious software engineer with a passion for learning new technologies. I am a team player and I enjoy working with others. I am always looking for new challenges and I naturally take on responsibility.\\

		I am looking for a new adventure, which allows me to travel outside of little Denmark. I am interested in a lot of different aspects of software development, and I would not mind taking on a role of a mentor. \\

		If you choose to hire me, you will get a hardworking colleague who naturally contributes to the team. You can expect a resource that works tirelessly to achieve the best possible result and who is eager to take on new challenges.\\
		}

		%----------------------------------------------------------------------------------------
		%	WORK EXPERIENCE
		%----------------------------------------------------------------------------------------
		\vfill\null
		\cvsection{WORK EXPERIENCE}

		\cvevent
		{Jun 23 - nu}
		{Full-stack Developer}
		{Trifork Smart Enterprise}
		{}
		{}%Arbejdsopgaver
		{\cvlist{
			\item Azure, Docker, Git, Kubernetes, React.js, Next.js, TypeScript
		}}%Involverede teknologier
		{\cvlist{
			\item Lead the cloud migration of Banedanmark's enterprise solution, Mobile Kontrolskemaer (MKS).
		}}%Resultater

		\cvevent
		{Feb 22 - Jun 23}
		{CTO \& Co-Founder}
		{Instructr}
		{Instructr was a B2B Software-as-a-Service (SaaS) company where I built a white-label blended learning platform for sports schools such as ski schools and surf schools.}
		{\cvlist{
			\item Development of the platform (frontend, backend, DevOps og cloud)
			\item Project management and manageing 4 employees
		}}%Arbejdsopgaver
		{\cvlist{
			\item ASP.NET, React.js, React Native, OpenAPI (Swagger), Entity Framework Core, Azure, Git, TypeScript
		}}%Involverede teknologier
		{\cvlist{
			\item Fully functional platform with multiple environments and paying customers using the production system
			\item Winner of several soft-funding competitions: Innofounder (840.000 DKK) and CBS Startup Competition (70.000 DKK)
		}}%Resultater

		\cvevent
		{Jun 21 - Jun 22}
		{Software Developer}
		% {Studenterprogrammør}
		{Trifork}
		{I developed new software for our customers. This includes everything from small automation tools to complete backend and frontend solutions.}
		{\cvlist{
				\item Domain modelling of an HR-datasystem both backend and database
				\item API design and implementation
				\item Development of a welding emulator in Swift
			}}%Arbejdsopgaver
		{\cvlist{
				\item ASP.NET, React.js, React Native, OpenAPI (Swagger), Entity Framework Core, MS SQL, Azure, Docker, Git, Swift, Kotlin
			}}%Involverede teknologier
		{}%Resultater

		% \cvevent
		% {Maj 21 - Jun 22}
		% {Mentor}
		% {Aalborg Universitet}
		% {Mentor for en softwareenginiørstuderende. Jeg hjalp den studerende med at forstå stoffet i kurserne samt at finde sig til rette socialt.}
		% {}%Arbejdsopgaver
		% {}%Involverede teknologier
		% {\cvlist{
		% 		\item Min mentee havde ambitioner om at blive den dygtigste software ingeniør, så det gjorde jeg mit bedste for at hjælpe ham med. Det har resulteret i flere topkarakterer, som han har været meget stolt af.
		% 	}}%Resultater

		\cvevent
		{Sep 20 - Jun 21}
		{Teaching Assistent}
		{Aalborg University}
		{Teaching Assistent in the courses \textbf{Imperative Programming} and \textbf{Algorithms and Data Structures}.}
		{\cvlist{
				\item Assisted the university students with their exercises after the lecture
				\item Corrected the student's assignments and provided them with constructive feedback
				\\\\\textit{A recommendation from the lecturer can be sent upon request}
			}}%Arbejdsopgaver
		{\cvlist{
				\item C (Programming language)
			}}%Involverede teknologier
		{}%Resultater

		% \cvevent
		% {Jul 20 - Nov 20}
		% {Rusinstruktør}
		% {Aalborg Universitet}
		% {Stod for at planlægge samt udføre sociale studiestartsarrangementer for de nye studerende på softwarestudiet.}
		% {\cvlist {
		% 		\item Var primus motor på streamingen af LAN-begivenheden, hvilket indebar at streame meddelelser og gameplay hjem til de studerende.
		% 		\item Uddelegerede opgaver til de 6 andre medlemmer af stream-teamet.
		% 	}}%Arbejdsopgaver
		% {}%Involverede teknologier
		% {\cvlist{
		% 		\item Forud for begivenheden formåede jeg at indhente sponsorater, som hver især havde en værdi på op til 10.000 kr. Sponsorerne var blandt andre: CEGO, Red Bull, BankData og PROSA. Pengene blev brugt på at gøre begivenheden unik med goodie bags til alle deltagere samt præmier til turneringsvinderene.
		% 	}}%Resultater

		% \cvevent
		% {Maj 20 - Jun 20}
		% {Programmeringsunderviser}
		% {Silkeborg Ungdomsskole}
		% {Introducerede en gruppe unge elever for basal webprogrammering.}
		% {\cvlist {
		% 		\item Planlagde og udførte selv undervisningen igennem hele forløbet.
		% 	}}%Arbejdsopgaver
		% {\cvlist {
		% 		\item HTML, CSS og JavaScript.
		% 		\item Discord (Undervisningen foregik virtuelt pga. coronapandemien).
		% 	}}%Involverede teknologier
		% {}%Resultater

		% \cvevent
		% 	{Nov 18 - Feb 19}
		% 	{Skiinstruktør}
		% 	{Niseko Village Snow School}
		% 	{Instruerede gæster i hvordan man står på ski. Arbejdede stort set hver dag hele sæsonen.}
		% 	{\cvlist {
		% 	    \item Instruktion af alle aldersgrupper og niveauer.
		% 	    \item Snerydning om morgenen.
		% 	    \item Opsætning af børneområde.
		% 	    \\\\\textit{Skiskolens vurdering af min indsats kan tilsendes, hvis det ønskes.}
		% 	}}%Arbejdsopgaver
		% 	{}%Involverede teknologier
		% 	{}%Resultater

		% \cvevent
		% 	{Jan 17 - Nov 18}
		% 	{Lærervikar}
		% 	{Dybkærskolen}
		% 	{Underviste børn og unge på folkeskoleniveau. Blev gradvist kaldt ind oftere og endte med at blive kaldt ind hver dag det sidste år af min ansættelse.}
		% 	{\cvlist {
		% 	    \item Underviste i alle fag på alle klassetrin.
		% 	    \item Faste lektioner i en periode på et halvt år i en modtagerklasse, hvor jeg udarbejdede mit eget. undervisningsmateriale\\\textbf{Fag:} Kulturfag.
		% 	    \item Faste lektioner i en periode på et halvt år i en 3. klasse.\\\textbf{Fag:} Musik, bibliotek og lektielæsning.
		% 	    \item Påtog mig det fulde ansvar og sikrede dermed en god kvalitet af undervisningen på trods af lærerens fravær.
		% 	    \\\\\textit{Udtalelse fra skolen kan tilsendes, hvis det ønskes.}
		% 	}}%Arbejdsopgaver
		% 	{}%Involverede teknologier
		% 	{}%Resultater

		% \cvevent
		% 	{Aug 17 - Nov 18}
		% 	{Pædagogmedhjælper}
		% 	{Dybkærklubben}
		% 	{Arbejdede både i klubbens fritidsafdeling og ungdomsafdeling. Havde fleksible arbejdstider, men arbejdede gennemsnitligt 18 timer ugentligt.}
		% 	{\cvlist {
		% 	    \item Ansvarlig for al IT i klubben (Playstation, PCer, Wii mv.).
		% 	    \item Ansvarlig for gymnastiksalen.
		% 	    \\\\\textit{Reference fra klubben kan tilsendes, hvis det ønskes.}
		% 	}}%Arbejdsopgaver
		% 	{}%Involverede teknologier
		% 	{}%Resultater

		% \cvevent
		% 	{Sep 17 - Sep 18}
		% 	{Pædagogmedhjælper}
		% 	{Myretuen SFO}
		% 	{Vikarierede i tilfælde af pædagogernes fravær, samt hvis klubben manglede mandskab til udflugter.}
		% 	{}%Arbejdsopgaver
		% 	{}%Involverede teknologier
		% 	{}%Resultater

		% \cvevent
		% 	{Aug 17 - Aug 18}
		% 	{Lektiehjælper}
		% 	{Privat}
		% 	{Hjalp tre børn/unge, som aldersmæssigt spændte fra 4. klasse til 1.g, med deres lektier.}
		% 	{}%Arbejdsopgaver
		% 	{}%Involverede teknologier
		% 	{}%Resultater

		% \cvevent
		% 	{Aug 14 - Jun 18}
		% 	{Parkourtræner}
		% 	{Silkeborg Ungdomsskole}
		% 	{Trænede op til fire hold om ugen 2 timer pr. hold.}
		% 	{\cvlist {
		% 	    \item Planlagde og udførte undervisning på ugentlig basis.
		% 	    \item Planlagde og udførte parkourture til flere destinationer i Danmark.
		% 	}}%Arbejdsopgaver
		% 	{}%Involverede teknologier
		% 	{\cvlist {
		% 	    \item Boostede parkourmiljøet i Silkeborg. I min tid som instruktør gik vi fra et ugentligt hold i Silkeborg Kommune til fire.
		% 	}}%Resultater

		% \cvevent
		% 	{Somrene 15 \& 17}
		% 	{Delikatesseassistent (15)\hfill \break Salgsassistent (17)}
		% 	{Føtex Ebeltoft}
		% 	{Sommerferiejob i Føtex Ebeltoft, hvor jeg hjalp butikken pga. det ekstra pres i sommerferien.}
		% 	{}%Arbejdsopgaver
		% 	{}%Involverede teknologier
		% 	{}%Resultater

		% \cvevent
		% 	{Sep 14 - Maj 15}
		% 	{Parkourtræner}
		% 	{Ungdomsskolen Favrskov}
		% 	{Trænede et hold i Hinnerup en gang om ugen i 2 timer.}
		% 	{\cvlist {
		% 	    \item Planlagde og udførte undervisning på ugentlig basis.
		% 	    \item Samlede en del af holdet op i Hammel og kørte dem i minibus til Hinnerup, så de også kunne deltage i træningen.
		% 	}}%Arbejdsopgaver
		% 	{}%Involverede teknologier
		% 	{}%Resultater

		% \cvevent
		% 	{Sep 11 - Sep 14}
		% 	{Servicemedarbejder}
		% 	{Føtex Nørrevænget}
		% 	{Servicemedarbejder i kolonialafdelingen.}
		% 	{}%Arbejdsopgaver
		% 	{}%Involverede teknologier
		% 	{}%Resultater

		% \cvevent
		% 	{Sep 08 - Jun 12}
		% 	{Trommelærer}
		% 	{Privat}
		% 	{Trommelærer for fire børn på ca. 9 år.}
		% 	{\cvlist {
		% 	    \item Planlagde og udførte undervisning på ugentlig basis.
		% 	    \item Udarbejdede mit eget undervisningsmateriale.
		% 	}}%Arbejdsopgaver
		% 	{}%Involverede teknologier
		% 	{}%Resultater

		%----------------------------------------------------------------------------------------
		%	FRIVILLIGE GØREMÅL
		%----------------------------------------------------------------------------------------
		% \vfill\null
		\cvsection{VOLUNTEERING}

		\cvevent
		{Dec 20 - Jun 22}
		{Chairman}
		{ABK-Net}
		{Chairman of the internet association on Arbejderbevægelsens Kollegium (ABK), where I lived.\\
			We were responsible for all 198 apartment's internet, ABK's home page as well as the ABK's mail service. This included handling the security and maintenance of the systems.\\\\
			GitLab: \url{https://gitlab.com/abk-aalborg}\\
		}
		{}%Arbejdsopgaver
		{\cvlist{
				\item Hugo, Netlify CMS, NixOS, Grafana
			}}%Involverede teknologier
		{\cvlist{
				\item ABK's home page: \url{https://abk-aalborg.dk/}
				\item Developed a new network setup from the bottom up, upgrading all of the residents of ABK's internet connection from 100 Mbit to 1 Gbit.
			}}%Resultater

		% \cvevent
		% {Aug 20 - Jun 22}
		% {Bestyrelsesmedlem}
		% {ABK-Net}
		% {}
		% {}%Arbejdsopgaver
		% {}%Involverede teknologier
		% {}%Resultater

		% \cvevent
		% {Sep 20 - Nov 21}
		% {Tutor}
		% {Aalborg Universitet}
		% {Sørgede for en god studiestart for de nye studerende på softwarestudiet.}
		% {}%Arbejdsopgaver
		% {}%Involverede teknologier
		% {}%Resultater

		% \cvevent
		% {Oct 20 - nu}
		% {Bestyrelsesmedlem}
		% {ABK-Beboerråd}
		% {Aktivt medlem af beboerrådet på Arbejdserbevægelsens Kollegium, hvor jeg bor.\\
		% 	Læs mere her: \url{https://abk-aalborg.dk/page/board/council/}}
		% {}%Arbejdsopgaver
		% {}%Involverede teknologier
		% {}%Resultater

		% \cvevent
		% 	{Feb 18 - nu}
		% 	{Skiinstruktør}
		% 	{Silkeborg Skiklub}
		% 	{Instruerer på klubbens årlige skitur i uge 7.}
		% 	{}%Arbejdsopgaver
		% 	{}%Involverede teknologier
		% 	{}%Resultater

		% \cvevent
		% 	{Feb 20 - nu}
		% 	{Skiinstruktør}
		% 	{Skiklubben Harreskov}
		% 	{Instruerer på ture, når det passer med studiet.}
		% 	{}%Arbejdsopgaver
		% 	{}%Involverede teknologier
		% 	{}%Resultater

		% \cvevent
		% 	{Aug 16 - Sep 16}
		% 	{Engelsklærer}
		% 	{Love Volunteers}
		% 	{Arbejdede i en kinesisk børnehave i 6 uger.
		% 	\\\\\textit{Min koordinators vurdering af min indsats kan tilsendes, hvis det ønskes.}}
		% 	{}%Arbejdsopgaver
		% 	{}%Involverede teknologier
		% 	{}%Resultater

		% \cvevent
		% {Dec 13 - Dec 15}
		% {Formand}
		% {Silkeborg Gymnasiums IKT-udvalg}
		% {Formand for Silkeborg Gymnasium Elevråds Informations- og Kommunikationsteknologiske udvalg.}
		% {}%Arbejdsopgaver
		% {}%Involverede teknologier
		% {}%Resultater

		% \cvevent
		% 	{Okt 13 - Jun 16}
		% 	{Elevrådsrepræsentant}
		% 	{Silkeborg Gymnasiums elevråd}
		% 	{Deltog aktivt som elevrådsrepræsentant i Silkeborg Gymnasiums elevråd.}
		% 	{}%Arbejdsopgaver
		% 	{}%Involverede teknologier
		% 	{}%Resultater

		% \cvevent
		% 	{Sep 13 - Maj 14}
		% 	{Parkourtræner}
		% 	{Silkeborg Ungdomsskole}
		% 	{Trænede et parkourhold en gang om ugen i 2 timer.
		% 	\\\\\textit{Min leders vurdering af min indsats kan tilsendes, hvis det ønskes.}}
		% 	{}%Arbejdsopgaver
		% 	{}%Involverede teknologier
		% 	{}%Resultater

		% hotfixes to create fake-space to ensure the whole height is used
		% \mbox{}
		% \vfill
		% \mbox{}
		% \vfill
		% \mbox{}
		% \vfill
		% \mbox{}
	\end{rightcolumn}
\end{paracol}
\end{document}

